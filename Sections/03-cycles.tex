%*----------- SLIDE -------------------------------------------------------------
\begin{frame}[t]{Ciclo Ingênuo}
    
    \begin{columns}
        \column{.01\textwidth}
        \column{.5\textwidth}
            \centering
            \includegraphics[width=.55\textwidth, trim= 0 0 0 0, clip]{ciclo1.png}
        \column{.6\textwidth}
            \centering
            Fazer uma busca numa base acadêmica e exportar no formato \emph{.bib}\\

            \vspace*{0.2cm}
            \includegraphics[width=.55\textwidth, trim= 0 0 0 0, clip]{bases.png}
    \end{columns}

%*----------- notes
    \note[item]{Notes can help you to remember important information. Turn on the notes option.}
\end{frame}
%-
%*----------- SLIDE -------------------------------------------------------------
\begin{frame}[t]{Ciclo Otimizado}
    
    \begin{columns}
        \column{.01\textwidth}
        \column{.5\textwidth}
            \centering
            \includegraphics[width=.55\textwidth, trim= 0 0 0 0, clip]{ciclo2.png}
        \column{.6\textwidth}
            \centering
            Testar a nova \emph{string}, e exportar os dados otimizado no formato \emph{.bib}\\

            \vspace*{0.2cm}
            \includegraphics[width=.85\textwidth, trim= 0 0 0 0, clip]{novabusca.png}
    \end{columns}

%*----------- notes
    \note[item]{Notes can help you to remember important information. Turn on the notes option.}
\end{frame}
%-
%*----------- SLIDE -------------------------------------------------------------
\begin{frame}[c]{Ciclo Otimizado}
    \framesubtitle{aplicando o RevTools}
    \begin{columns}
        \column{.01\textwidth}
        \column{.5\textwidth}
            \centering
            \includegraphics[width=.95\textwidth, trim= 0 0 0 0, clip]{revtools.png}
        \column{.5\textwidth}
    
            O \emph{RevTools} é um pacote do R utilizado para apoiar os pesquisadores que trabalham em projetos de síntese de evidências. \\
            Propicia a visualização baseada em padrões bibliográficos.\\
            Neste método, ele irá promover a exclusão dos artigos que não tem referência com o devido estudo.

    \end{columns}

%*----------- notes
    \note[item]{Notes can help you to remember important information. Turn on the notes option.}
\end{frame}
%-
%*----------- SLIDE -------------------------------------------------------------
\begin{frame}[c]{Ciclo Impacto}
    \begin{columns}
        \column{.01\textwidth}
        \column{.5\textwidth}
            \centering
            \includegraphics[width=.95\textwidth, trim= 0 0 0 0, clip]{ciclo3.png}
        \column{.5\textwidth}
            Alguns pontos importantes nesta fase:
            \begin{itemize}
                \item Analisar o impacto do autor.
                \item Analisar o impacto do autor por total de citações.
                \item Selecionar os artigos principais dos cinco mais impactantes autores.
            \end{itemize}
        
    \end{columns}

%*----------- notes
    \note[item]{Notes can help you to remember important information. Turn on the notes option.}
\end{frame}
%-
%*----------- SLIDE -------------------------------------------------------------
\begin{frame}[c]{Ciclo Produção}
    \begin{columns}
        \column{.01\textwidth}
        \column{.5\textwidth}
            \centering
            \includegraphics[width=.95\textwidth, trim= 0 0 0 0, clip]{ciclo4.png}
        \column{.5\textwidth}
            Gerar documentos de análise.

            \centering
            \begin{figure}
                     \includegraphics[width=0.7\textwidth]{mendeley.png}
                     \hfill
                     \includegraphics[width=0.7\textwidth]{cmaptools.png}
            \end{figure}
        
    \end{columns}

%*----------- notes
    \note[item]{Notes can help you to remember important information. Turn on the notes option.}
\end{frame}
%-
%*----------- SLIDE -------------------------------------------------------------
{
\setbeamertemplate{background}
{\includegraphics[width =\the\paperwidth, clip, trim = 0 0 0 100]{nick-morrison-FHnnjk1Yj7Y-unsplash.jpg}}
%*----------- SLIDE -------------------------------------------------------------
\begin{frame}[t]{} 
\end{frame}
}
%-
%*----------- SLIDE -------------------------------------------------------------
\begin{frame}[c]{}
    \centering
    \includegraphics[width=.65\textwidth, trim= 0 0 0 0, clip]{github.png}
    
    \vspace*{0.3cm}
    \href{https://github.com/Brazilian-Institute-of-Robotics/bir-mini-method-bili}{link-do-repositório}
%*----------- notes
    \note[item]{Notes can help you to remember important information. Turn on the notes option.}
\end{frame}
%-